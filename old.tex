Contour operations are implemented in methods, taking Common Lisp
object systems (CLOS) multiple order. So a method like \texttt{rotate}
behaves in a different way depending on the first parameter. In code
in figure \ref{fig:code-methods} there are two methods, the first to
\code{contorno-duracao}, and the second to \code{contorno-simples}.

%% FIXME: traduzir termos do goiaba
\begin{figure*}
\begin{verbatim}
(defmethod rotacionar ((objeto contorno-duracao) &optional (fator 1))
  (if (> fator (length (pontos objeto)))
      (let ((x (mapcar #'ponto-x (pontos objeto)))
            (y (mapcar #'ponto-y (pontos objeto))))
        (make-contorno-duracao (mapcar #'make-ponto x
           (append (subseq y fator) (subseq y 0 fator)))))))

(defmethod rotacionar ((objeto contorno-simples) &optional (fator 1))
  (make-contorno-simples (append (subseq (pontos objeto) fator)
                                 (subseq (pontos objeto) 0 fator))))
\end{verbatim}
  \caption{Methods}
  \label{fig:code-methods}
\end{figure*}

%%% Local Variables: 
%%% mode: latex
%%% TeX-master: "goiaba-contour-processor"
%%% End: 

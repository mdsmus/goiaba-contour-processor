% brainstorm:

% * citar fundamento em teorias de contornos e pouco uso em composição
% * exemplos das necessidades de operações na composição (peça do mestrado)
% * desenvolvimento do goiaba (para que serve, exemplo de código e de
% saída)
% * trabalhos futuros

\section{Introduction}
\label{sec:introduction}

\section{Contour applications in composition}
\label{sec:cont-appl-comp}

\section{Goiaba}
\label{sec:goiaba}

\goiaba{} is a software developed in Common Lisp \cite{graham94:lisp},
compiled with Steel Bank Common Lisp (SBCL) \cite{team07:sbcl}, and
bottom up metodology.

Bottom up methodology is related to the development direction, from
simple subsystems to the entire program. This methodology is suitable
to programs that progressively have complexity increased. Programs
like \TeX{}\footnote{\url{www.tug.org}} and X
Windows\footnote{\url{www.x.org}} are developed in this way. This
metodology has good advantages as the possibility to make smaller and
smarter programs, to reuse subsystems functions in others subsystems,
and to simplify and clear programmer ideas \cite{graham94:lisp}.

In \goiaba{}, contours are symbolic represented in two ways: as simple
contours and duration-included contours. Simple contours represents
only contour element values: \code{(5 9 6)}. Duration-included
contours represents also the time when the element occur: \code{((0
  5)(1 9)(2 6))}. Despite \goiaba{} provides a codification that
includes time influence in contour, we don't consider time. It's a
future work (see section \ref{sec:future-work}).

\goiaba{} has \texttt{point}, \texttt{simple-contour} and
\texttt{duration-contour} classes. The \texttt{point} class defines
pitches in time as (x, y), cartesian elements, x as time and y as
pitch. The \texttt{duration-contour} class defines a contour as a
points list, like ((x, y) (z, w)), and the \texttt{simple-contour}
class defines contours only as pitches, like (y w).

Some read macros were defined to simplify instances making easier. For
instance, to make \texttt{point} instance, it's possible to do
\code{(make-instance 'ponto :x 0 :y 3)} or just \code{\#p(0 3)}, with
read macro. It's possible also to create \texttt{duration-contour}
with two points in \code{\#d(\#p(0 3) \#p(1 4))}.

Contour operations are implemented in methods, taking Common Lisp
object systems (CLOS) multiple order. So a method like \texttt{rotate}
behaves in a different way depending on the first parameter. In
following code there are two methods, the first to
\code{duration-contour}, and the second to \code{simple-contour}.

%% FIXME: insert code


%% FIXME: update operations
\goiaba{} has many operations to process contours, like inversion,
retrogradation and rotation, contour reduction \cite{adams76:melodic},
contour class, contour adjacency series, contour adjacency series
vector, contour interval, contour interval array, contour class vector
I and II \cite{friedmann85:methodology}, and comparison matrix
\cite{morris93:directions}.

Finally \goiaba{} uses Cl-pdf
library\footnote{\url{www.cliki.net/CL-PDF}} to plot contours,
allowing easy contour operations visualization. For instance, the
follow code generates a graph with original contour, retrogradation,
inversion, and rotation (fig. \ref{fig:operacoes}). In this example we
define \code{Z} contour, output file, graphic dimensions, operations,
and colors to be output in graphic.

%% FIXME: insert code

%% FIXME: insert figure
\begin{figure}
  \centering
  \caption{Z(0 5 3 4 1 3) contour operations}
  \label{fig:operacoes}
\end{figure}

\section{Future work}
\label{sec:future-work}

%%% Local Variables: 
%%% mode: latex
%%% TeX-master: "goiaba-contour-processor"
%%% End: 
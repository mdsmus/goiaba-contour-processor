% usar para termos estrangeiros
\newcommand{\eng}[1]{\textit{#1}}
% \newcommand{\eg}[1]{\textit{\gls{#1}}}
\newcommand{\eg}[1]{\textit{#1}}

\newcommand{\opus}[1]{\textit{#1}}
\newcommand{\tr}[1]{\textit{#1}}

\newcommand{\contorno}[1]{$\langle #1 \rangle$}
\newcommand{\contpr}{P(5 3 4 1 2 0)}
\newcommand{\obra}{\textit{Em torno da romã}}
\newcommand{\goiaba}{\opus{Goiaba}}
\newcommand{\code}[1]{\texttt{#1}}

\newcommand{\citacaoinline}[4]{
  ``#1''\footnote{
    \selectlanguage{brazil}
    ``{#2}''.}
  \selectlanguage{brazil}
  \cite[#3]{#4}.
}

\newcounter{notecounter}

\newcommand{\note}[1]{
  \addtocounter{notecounter}{1}
  \textcolor{red}{[note \arabic{notecounter}: #1]}
}

\newcommand{\cinzaa}{
  \multicolumn{1}{>{\columncolor[gray]{.5}}c}{}
}

\newcommand{\cinzab}{
  \multicolumn{1}{>{\columncolor[gray]{.7}}c}{}
}

\newcommand{\music}[1]{\textit{\textbf{#1}}}

\makeatletter
\newbox\sf@box
\newenvironment{SubFloat}[2][]%
{\def\sf@one{#1}%
\def\sf@two{#2}%
\setbox\sf@box\hbox
\bgroup}%
{ \egroup
\ifx\@empty\sf@two\@empty\relax
\def\sf@two{\@empty}
\fi
\ifx\@empty\sf@one\@empty\relax
\subfloat[\sf@two]{\box\sf@box}%
\else
\subfloat[\sf@one][\sf@two]{\box\sf@box}%
\fi}
\makeatother
